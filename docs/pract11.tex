\documentclass{beamer}
\usepackage[spanish]{babel}
\usepackage[utf8]{inputenc}
\usepackage {graphicx}

\title {El número $\pi$}
\author[Maria]{María Baeza López}
\institute{Facultad de Matemáticas}
\date[24/04/2014]{La Laguna, 24 de Abril de 2014}
\usetheme{Madrid}
\definecolor{MiVioleta}{RGB}{122,59,122}
\definecolor{MiAzul}{RGB}{0,88,147}
\definecolor{MiGris}{RGB}{56,61,66}
\setbeamercolor*{palette primary}{use=structure,fg=white,bg=MiVioleta}
\setbeamercolor*{palette secundary}{use=structure,fg=white,bg=MiAzul}
\setbeamercolor*{palette tertiary}{use=structure,fg=white,bg=MiGris}
\begin{document}
\begin{frame} 
\titlepage 
\end {frame}
\begin{frame} 
\frametitle{Indice}
\tableofcontents[pausesections]
\end {frame}
\section {Introducción}
\begin{frame} 
\frametitle{Introducción}
El número $\pi$ es posiblemente la constante numérica más estudiada a lo largo de la historia. Su entorno, aunque en apariencia sólo a nivel matemático, a ha 
trascendido las fronteras de esta disciplina y es así como ha suscitado el interés de hombres en diversas áreas del conocimiento. La revisión de su desarrollo 
histórico es, por tanto, una combinación amena de aspectos científicos, anecdóticos y culturales.
\end {frame}
\section {Marco general}
\begin{frame} 
\frametitle{General}
La historia del número $\pi$ puede dividirse en tres períodos claramente establecidos, los cuales se diferencian entre sí por aspectos relacionados con el método,
propósitos inmediatos y las herramientas científicas e intelectuales disponibles:
\begin{itemize}
\item Primer Periodo: Las primeras trazas de la determinación de $\pi$ son encontradas en diferentes papiros de gran antigüedad, que a manera de catálogos incluían en las modalidades de 
escritura de las respectivas épocas, grupos de problemas y su correspondiente solución, reflejo del estado de las matemáticas de aquella cultura a la que pertenecen.
El más conocido de ellos es el llamado papiro de Rhind \pause
\item Segundo Periodo: El establecimiento de los fundamentos del cálculo diferencial e integral por parte de Newton y Leibniz durante la segunda mitad del siglo XVII, y los posteriores
desarrollos en esta área abren, como se mencionó previamente en el marco general, la segunda parte de esta historia.
\end{itemize}
\end {frame}
\begin{frame} 
\frametitle{General}
\begin{itemize}
\item Tercer Periodo:El tercer período de esta historia podría resumirse en las respuestas que se obtuvieron a la pregunta: ¿Cuál es el lugar de entre los números?. En efecto, los trabajos
realizados durante este período se enfocaron principalmente a la investigación de la real naturaleza de $\pi$.
Debido a la estrecha relación de este número con la constante e , la base de los logaritmos naturales, la investigación de los dos números fue llevada casi de manera 
simultánea.
\end{itemize}
\end {frame}
\section {Fórmulas}
\begin{frame}
\frametitle{Fórmulas}
El número $\pi$ es infinito, las primeras 2 cifras son : $\pi \approx 3.14$

El número $\pi$ las primeras 4 cifras son : $\pi \approx 3.1415$

El número $\pi$ las primeras 6 cifras son : $\pi \approx 3.141592$

El número $\pi$ las primeras 8 cifras son : $\pi \approx 3.14159265$

Podemos seguir haciendo aproximaciones de $\pi$ gracias a la fórmula:
\begin{block}{Aproximación}
$\pi \approx \sum^{n}_{i=1} f(x_i)$
\end{block}
\end {frame}
\begin{frame} 
\frametitle{Bibliografía}
\begin{thebibliography}
\beamertermplatebookbibitems
\bibitem[Guia Docente,2013]{guia}
Guía Docente (Año 2013)
{\small $https://www.github.com/alu0100530964.git$}
\bibitem[El n{\'u}mero PI y su historia(2011)]{Reif Acherman, Simon}
 El n{\'u}mero PI y su historia(2011)
{\small $https://www.ull.es$}
\end{thebibliography}
\end {frame}
\end{document}